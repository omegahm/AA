%
%  Created by Kasper Nybo Hansen on 2011-04-28.
%  Copyright (c) 2011 Kasper Nybo Hansen. All rights reserved.
%
%
\documentclass[10pt]{article}

\RequirePackage{nybohansenPreamble}

\newcommand{\authorName}{Mads Ohm Larsen \& Kasper Nybo Hansen}
\newcommand{\authorEmail}{\{omega, nybo\}@diku.dk}
\newcommand{\titleName}{Project 1}
\newcommand{\courseName}{Advanced Algorithms 2011}

\author{\authorName \\\texttt{\small{\authorEmail}}}
\title{\textsc{\titleName \\ \courseName}}
% \date{}
\makeindex

\begin{document}
\maketitle    

\section*{Question 1.1} % (fold)
\label{sec:question_1_1}


A flow network $G = (V,E)$ is a \emph{directed} graph in which each edge $(u,v) \in E$ has a nonnegative capacity $c(u,v)\geq 0$ 0. We further require that if $E$ contains an edge $(u,v)$, then there is no edge $(v,u)$ in the reverse direction\cite{Cormen}.

The supplied graph $G$ is not a flow network, since it doesn't comply with the definition of a flow network since it is undirected. 

% section question_1_1 (end)

\section*{Question 1.2} % (fold)
\label{sec:question_1_2}

In the following we assume that the given capacity of each street segment is the capacity in each direction of the street. 

In order to convert $G$ into a flow network we need to replace all existing edges by two `lanes'. For each edge going between the vertices $u$ and $v$ and having the capacity $c$ we define a new vertex $w$. Furthermore we define two directed edges $(u,w)$ and $(w,v)$, both with capacity $c$. We also introduce $(v,u)$ as a directed edge with capacity $c$. The result is a directed graph $G'$ that is a flow network with vertices $0-6$ being sources and vertices $26-29$ being sink. 
% section question_1_2 (end)


\section*{Question 1.3} % (fold)
\label{sec:question_1_3}

% section question_1_3 (end)


\section*{Question 1.4} % (fold)
\label{sec:question_1_4}

% section question_1_4 (end)


\bibliographystyle{abbrv}
\bibliography{bibliography}

\end{document}                      



